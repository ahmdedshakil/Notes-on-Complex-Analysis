\documentclass{article}
\usepackage[homework]{ahmeds}

\initialize{}

\begin{document}

\start{Ahmed Shakil}{Math H185}{Midterm 2 Review}

\tableofcontents

\newpage

\section{Tools and Tricks}
\begin{thrm}{Weierstrass's Approximation Theorem}{}
Any continuous function \( f(x) \) on \( [0,1] \) can be uniformly approximated by polynomials, meaning that one can find a sequence of polynomials \( p_{n} (x) \) converging uniformly to \( f(x) \) on \( [0,1] \).
\end{thrm}

\begin{thrm}{}{}
If \( f:\mathbb{C} \to \mathbb{C}  \) is holomorphic and non-constant, then \( f(\mathbb{C} ) \) is dense in \( \mathbb{C}  \). 
\end{thrm}


\newpage

\section{Analytic Functions}
\begin{thrm}{Dominated Convergence}{}
Suppose there exist a function \( G(x) \) such that \( \left\lvert F_n(x) \right\rvert < \left\lvert G(x) \right\rvert \ \forall \, n,x \) and \( \int \left\lvert G(x) \right\rvert dx< \infty  \), then
\[
    \lim_{n \to \infty} \int F_{n} (x) dx = \int \lim_{n \to \infty} F_{n} (x) dx.
\]
\end{thrm}  
 Many times a constant can serve in the place of such \( G(x) \).


As a reminder we note the following results again. 
\begin{thrm}{Cauchy's formula}{}   
Suppose we have \( f: \Omega \subset \mathbb{C}  \to \mathbb{C}  \) which is defined on an open set as the domain, and is holomorphic. Let \( z_0 \in \Omega  \) and \( r > 0 \) such that \( B_r(z_0) \subset \Omega  \). Then for all \( z \in B_ r(z_0), \) we have 
\[
    f(z) = \frac{1}{2\pi i}\int _{\partial B_r(z_0)}\frac{f(w)}{w - z}dw. 
\]

\end{thrm}

\begin{cor}{Infinite differentiability}{}
As a result of Cauchy's formula we have that wherever \( f \) is holomorphic it is infinitely differentiable. 

We can differentiate Cauchy's formula to derive the following more general formula for higher order derivatives:

\begin{align*}
    \boxed{f^{(n)}(z) = \frac{n!}{2\pi i}\int _{\partial B_r(z_0) } \frac{f(w)}{(w - z)^{n+ 1} } dw }
\end{align*}

\end{cor}


We use this formula's to derive a bound on the \( n^{\text{th} }  \) derivative of \( f \). 

\begin{thrm}{Cauchy's Inequality}{}
Let \( f: U \to \mathbb{C} \) be a holomorphic function and take some \( \overline{B_r(z)} \subset U  \), then
\[
    \left\lvert f^{(n)}(z) \right\rvert \leq \frac{n!}{r^n}\sup _{w\in \partial B_r (z)} \left\lvert f(w) \right\rvert .
\]
\tcbline

Apply the integral triangle inequality to Cauchy's Integral formula and the result follows.
\begin{proof}

    \begin{align*}
        \left\lvert f^{(n)}(z) \right\rvert &\leq \frac{n!}{2\pi i} \int_{\partial B_r(z)} \left\lvert \frac{f(w)}{(w - z)^{n+ 1} }  \right\rvert \left\lvert dw \right\rvert \\
        &\leq \frac{n!}{2\pi i}\frac{2\pi i r}{r^{n+ 1} }  \sup _{w \in \partial B_r(z)} f(w)\\
        &= \frac{n!}{r^n} \sup _{w \in \partial B_r(z)} f(w)
    \end{align*}
    
\end{proof}

\end{thrm}

\begin{defn}{Analytic function}{}
A function \( f: U \to \mathbb{C}  \) is analytic if \( \forall \, z_0 \in U,  \ \exists \, r>0 \) such that 
\[
    f(z) = \sum_{n= 0}^{\infty} a_{n} (z - z_0)^n  \ \ \  \forall z\in B_r(z_0).
\]
It is important to note that being analytic implies being holomorphic and vice-versa. 
\end{defn}

\begin{thrm}{Taylor representation}{}
If \( f \) is holomorphic on \( B_r(z_0) \subset U \), and we have that  
\( f(z) = f(z _0) + f^\prime (z_0)(z - z_0)+ \frac{1}{2!}f^{\prime\prime} (z_0)(z - z_0)^{2} + \dots  \) Then this series converges for \( z \in B_r(z_0) \). 

\tcbline

\begin{proof}
Follows from using Cauchy's inequality to bound the terms in the Taylor series in order to calculate a radius of convergence which ends up being \( \geq r \). 
\end{proof}

\end{thrm}


\begin{thrm}{Taylor Expansion of a Holomorphic Function}{}
If \( f \) is holomorphic \( B_r(z_0) \subset U \), then 
\[
    f(z) = f(z_0) + f^\prime (z_0)(z - z_0)+ \frac{f''(z_0)}{2!}(z - z_0)^{2} +\dots 
\]
\tcbline
\begin{proof}
We start with Cauchy's formula:

\begin{align*}
    f(z) = \frac{1}{2\pi i} \int _{\partial B_r(z)} \frac{f(w)}{w - z} dz.
\end{align*}


\[
    \frac{1}{w - z} = \frac{1}{(w - z_0) - (z - z_0)} = \frac{1}{w - z_0} \left( \frac{1}{1 - \frac{z - z_{0} }{w - z_0} } \right)    
\]
 
Note that since \( |z - z_0| < |w - z_0| \) we have \( \frac{|z - z_0|}{|w - z_0|} < 1 \).

So
\[
    \frac{1}{1 -\frac{z - z_0}{w - z_0} } = 1 + \frac{z - z_0}{w - z_0} + \left( \frac{z - z_0}{w - z_0} \right)^{2} + \dots  
\]

Hence, we now have:

\[
    \frac{1}{w - z} = \frac{1}{w - z_0}\left( 1 + \frac{z - z_0}{w - z_0} + \left( \frac{z - z_0}{w - z_0} \right)^{2} + \dots   \right)  
\]

\begin{align*}
    f(z) = \int _{\partial B_r(z_0)} \frac{f(w)}{w - z_0} + \frac{f(w)}{(w - z_0)^{2} }(z - z _0) + \dots   dw
\end{align*}

Using the Dominated Convergence Theorem, we can interchange the sum and integral, from there we can leverage Cauchy's integral formula:

\begin{align*}
    f(z) &= \frac{1}{2\pi i}\int _{B_r(z_0)}\frac{f(w)}{w - z_0}dw + \int _{B_r(z_0)}\frac{f(w)}{(w - z_0)^{2} }(z - z_0) dw+\dots \\
    &= f(z_0) + f'(z_0)(z - z_0)+ \frac{f''(z_0)}{2!}(z - z_0)^{2} +\dots 
\end{align*}
This is the Taylor expansion, hence we are done. 
\end{proof}


\end{thrm}
An important slogan to remember is that holomorphic functions are automatically analytic.

\begin{defn}{Entire function}{}
A priori we know what it means for a function to be bounded. We say a function is entire if it is defined and holomorphic on all the complex numbers. 
\end{defn}

\begin{thrm}{Liouville's Theorem}{}
If \( f: \mathbb{C} \to \mathbb{C}  \) is entire, and bounded, then \( f \) is constant.
\tcbline
\begin{proof}
Apply Cauchy's inequality to \( B_r(z_0) \),
\[
    \left\lvert f^\prime (z_0) \right\rvert\leq \frac{1}{r} \sup _{z \in \partial B_r(z_0)} \left\lvert f(z) \right\rvert  \leq A.
\]
Taking \( r \to \infty  \), we get \( f^\prime (z_0) = 0 \). 
\end{proof}

\end{thrm}


\begin{thrm}{Fundamental Theorem of Algebra}{}
Every nonconstant polynomial
\[
    f(z) = a_{n} z^n + \dots + a_{1}z + a_0 
\]
has a root. 

\tcbline

\begin{proof}
Suppose not for the sake of contradiction. Since \( f(z) \) is entire and non-zero, \( \frac{1}{f(z)} \) is entire.  

 Indeed, 
\begin{align*}
    \frac{1}{f(z)} = \frac{1}{z^n} \left( \frac{1}{\frac{a_0}{z^n} + \dots + a_{n} } \right) 
\end{align*}
As \( z \to  \infty  \), then \( \frac{1}{f(z)} \to 0 \). This means for every \( \epsilon   \)  there exists an \( R \) such that \( \forall \ |z|>R: \  \left\lvert \frac{1}{f(z)} \right\rvert  < \epsilon  \). Now for the \( |z| \leq R \)  we have the following:
\[
    \left\lvert \frac{1}{f(z)} \right\rvert  \leq  \sup _{w \in \partial B_R(0)} |\frac{1}{f(w)}| < A \neq \infty .
\]

Now since \( \frac{1}{f(z)} \) is bounded and entire, by Liouville's theorem we have that \( \frac{1}{f(z)} \) is constant. However, this is a contradiction to our initial assumption, hence, we are done. 
\end{proof}

\end{thrm}
\begin{cor}{Number of Roots}{}
As a direct consequence of the previous theorem we have that a polynomial of degree \( n \) has \( n \) roots when counted with multiplicity. The proof goes by induction, where you first factor out a root and then continuously proceed to factor roots. 
\end{cor}

\subsection{Analytic Continuation}

\begin{thrm}{About the sparsity of zeroes.}{}
Suppose \( f: U \to \mathbb{C}  \) is holomorphic and  \( U \) is connected and open, and \( f \) vanishes on \( \left\{ w_{n}  \right\}  \subset U\) such that \( \lim_{n \to \infty} w_{n}  \) exists in \( U \). Given that \(\left\{ w_{n} \in U : f(w_{n} ) = 0\right\}  \) has a limit point in \( U \), then \( f \equiv 0 \).  

\tcbline 

\begin{proof}
Let \( z_0 = \lim_{n \to \infty} w_{n}  \in  U\). Since \( f \) is holomorphic at \( z_0 \) we have that \( f \) is analytic at \( z_0 \). 

\[
    \implies f(z) = \sum_{n= 0 }^{\infty} a_{n} (z - z_0)^n \ \ \ \forall z\in B_r(z_0)
\]
For some \( r\in \mathbb{R} _{>0} \). Suppose \( f \neq 0 \) on \( B_r(z_0) \). Then \( \exists  \) some smallest \( m \) such that \( a_{m} \neq 0  \),

\[
    f(z) = a_{m} (z - z_0)^m (1 + g(z)),
\]
where \( \lim_{z \to z_0} g(z) = 0 \). For \( \left\lvert z - z_0 \right\rvert  \) small enough, say \( <\delta  \), then \( \left\lvert 1 + g(z) \right\rvert  > \frac{1}{2}\), but then \( f(z) \neq 0  \) if \( z \in B_\delta (z_0) \setminus \left\{ z_0 \right\}   \). However, this is a contradiction based on the assumptions we laid out initially about the sequence \( \left\{ w_{n}  \right\}  \). Thus, \( f(z) = 0 \ \ \forall z \in B_r(z_0)  \). 

We are not done: we have to use that \( U \) is connected.

\begin{defn}{Connected Set}{}
A subset \( S \subset \mathbb{C}  \) is connected if whenever \( S = V_1 \cup V_2 \) then \( V_1 = \varnothing  \) or \( V_2 = \varnothing  \), where \( V_1 \) and \( V_2 \) are open subsets of \( S \). 
\end{defn}

\underline{Back to the Proof:}

Let \( \Omega = \left\{ w_{n} \in U : f(w_{n} ) = 0\right\}^-  \), i.e. the interior of the set. Now based on our previous work we know that \(\exists B_r(z_0) \subset \left\{ z \in U: f(z) = 0 \right\}   \), hence the interior is not empty. By definition the interior is open. Now we show that it is closed. Suppose we have a sequence \( \left\{ z_{n}  \right\}  \subset \Omega \)  such that \( \lim_{n \to \infty} z_{n} = z  \). Since (by our previous work) \( f \) vanishes in a ball around \( z \), we have \( z \in \Omega  \). Hence, \( \Omega  \) is closed. 

\[
    \text{So } U = \underbrace{\Omega }_{\text{open} } \cup   \underbrace{(U \setminus \Omega ) }_{\Omega \text{ closed } \implies  \text{ open} } .
\]
But by the definition of a connected set it follows that \( U = \Omega  \). 
\end{proof}

\end{thrm}

\begin{cor}{}{}
If \( f,g \) are holomorphic on connected \( U \subset \mathbb{C}  \) and \( f(z) = g(z) \) on a convergent sequence, then \( f = g \) on all of \( U \). 
\end{cor}

\begin{defn}{Analytic Continuation}{}
If \( f \) is holomorphic on \( U \subset V \subset \mathbb{C}  \) and \( g \) is holomorphic on \( V \subset \mathbb{C}  \), and \( g = f \)  on \( U \), then \( g  \) is an analytic continuation of \( f \). 
\end{defn}


\begin{exmp}{Riemann \( \zeta  \) function}{}
    \[
        \zeta (s) = \frac{1}{1^s} + \frac{1}{2^s} + \dots 
    \]

    We have that \( n^s = e^{s \log (n) }  \) makes sense for \( s \in \mathbb{C}  \) (note we are using the principal branch). 

    \[
        \left\lvert n^{x+ iy}  \right\rvert = n^x
    \]
    This implies that the zeta function converges for \( \Re (s) > 1 \). 

    Riemann proved that there is an analytic continuation of \( \zeta (s) \) to \( \mathbb{C} \setminus \left\{ 1 \right\}  \). 
\end{exmp}

\section{Analytic Continuation of \( \zeta (s) \) }

\begin{misc}{Proposition}{}
For \( \Re (s)>0 \),

\[
    \zeta (s) = \frac{s}{s - 1} - s \int _1^{\infty}  \frac{\left\{ x \right\} }{x^{s+ 1} } dx.  
\]

\end{misc}


\section{Sequences of Holomorphic Functions}

\begin{thrm}{Morera's Theorem}{}
If \( f \) is continuous on \( B_r(z_0) \subset U\) (where \( U \) is open) and \( \int _T f(z) dz = 0 \) for every \( T \subset B_r(z_0) \), then \( f \) is holomorphic on \( B_{r}(z_0)  \). Here \( T \) is a triangle. 
\end{thrm}

\begin{thrm}{}{}
If \( \left\{ f_{n}  \right\}_{n = 1}^{\infty}   \) sequence of holomorphic functions converges to \( f\) uniformly on compact subsets, then \( f \) is holomorphic. 
\tcbline
\begin{proof}
Suffices to show \( 0 = \int_T f(z) dz \) for all triangles \( T \subset U  \). 

\begin{align*}
    \int_T \lim_{n \to \infty} f_{n} (z) dz = \lim_{n \to \infty} \int _Tf_{n} (z) dz = 0
\end{align*}

We can do this since we can exchange limits and integrals for uniformly converging sequences of functions. 

\end{proof}

\end{thrm}

\begin{thrm}{}{}
    If a sequence \( \left\{ f_{n}  \right\}_{n = 1}^{\infty}   \) of holomorphic functions converges to \( f\) uniformly on compact subsets, then \( f_{n} ' \) converges uniformly to \( f' \) on compact subsets. 

    \tcbline

    \begin{proof}
    Let \( z \in U \) and \( r>0 \) such that \( \overline{B_{r} (z_0)}  \subset U\). Now based on Cauchy's Integral formula
    \[
        f_{n} '(z_0) = \frac{1}{2\pi i}\int _{\partial B_{r} (z_0)} \frac{f_{n} (z)}{(z - z_0)^2}dz.  
    \]
    So now:
    \[
        \lim_{n \to \infty} f_{n} '(z_0) = \lim_{n \to \infty}  \frac{1}{2\pi i}\int _{\partial B_{r} (z_0)} \frac{f_{n} (z)}{(z - z_0)^2}dz = \frac{1}{2\pi i }   \int _{\partial B_{r} (z_0)}\lim_{n \to \infty} \frac{f_{n} (z_0)}{(z - z_0)^2} dz =  \frac{1}{2\pi i}\int _{\partial B_{r} (z_0)} \frac{f (z)}{(z - z_0)^2}dz.
    \]
    \end{proof}
    
\end{thrm}
\subsection{Schwartz Reflection Principal}

The goal is to establish criterion for analytic continuation in a specific situation. Take a symmetric set \( U \) which is symmetric about the real axis. 

\begin{thrm}{Glueing holomorphic functions along a boundary}{}
If \( f^+ \) is holomorphic on \( U^+ \) and \( f^- \) is holomorphic on \( U^- \) and extend continuously to \( I \), then 
\begin{align*}
    f(z) = 
    \begin{cases}
        f^+(z) \ \  z \in U^+ \\ 
        f^-(z) \ \ z \in U^- \\
        f^+(z) = f^-(z) \ \ z \in I 
    \end{cases}
\end{align*}
is holomorphic. 
\tcbline
\begin{proof}
Using morera's theorem we can split triangles up into cases, and then we are done. 
\end{proof}

\end{thrm}
\begin{thrm}{Schwarz reflection principle}{}
If \( f^+ \) is holomorphic on \( U^+ \) and extends continuously to \( I \) and \( f(z) \in \mathbb{R}  \ \ \forall z\in I \) then \( \exists  \) analytic continuation of \( f \) to \( U \). 
\end{thrm}

\section{Singularities}

\begin{defn}{Singularity}{}
An isolated singularity of a function \( f \) is \( z_0 \in \mathbb{C}  \) such that \( f \) is defined on \( B_{r} (z_0)\setminus \left\{ z_0 \right\}  \) for some \( r>0 \).
\end{defn}

\begin{defn}{Zero}{}
A zero of a function \( f \) is a \( z_0 \in \mathbb{C}   \) such that \( f(z_0) = 0 \), we say \( f \) "vanishes" at \( z_0 \). 
\end{defn}
\begin{thrm}{}{}
If \( f \) holomorphic near \( z_0 \) and \( f \) is not identically zero near \( z_0 \), then \( z_0 \) is an isolated zero. 
\end{thrm}

\begin{thrm}{}{}
If \( f \) holomorphic near some root \( z_0 \) and \( f \) is not identically \( 0 \) near \( z_0 \), then \( z_0 \) is an isolated zero of \( f \), i.e., \( \exists r>0 \) such that \( f \) does not vanish at any \( z \in B_{r} (z_0) \setminus \left\{ z_{0}  \right\} \). This is the contrapositive of analytic continuation.
\end{thrm}

\begin{thrm}{}{}
Let \( f \) be holomorphic near some root \( z_0 \), then \( \exists  n\geq 1 \), and \( g \) holomorphic and nonvanishing near \( z_0 \) such that \( f(z) = (z - z_0)^n g(z) \ \ \forall z \in B_r(z_0), \text{ for some } r> 0 \). 
\tcbline
\begin{proof}
Since \( f \) is holomorphic near \( z_0 \), then we have 
\[
    f(z) = \sum_{k= 0}^{\infty} a_{k} (z - z_0)^k. 
\]
Let \( n \) be the minimal \( n \) such that \( a_{n} \neq 0 \). 
Then 
\[
  f(z) = (z - z_0)^n \underbrace{\sum_{k\geq n}^{\infty}   a_{k} (z - z_{0} )^{k- n} }_{g(z)}.
\] 
By assumption we see the \( g(z) \) is non vanishing on \( B_{r} (z_0) \) and still has a radius of convergence. 
\end{proof}

\end{thrm}
\begin{defn}{Merorphic function}{}
Suppose \( f \) has an isolated singularity at \( z_0 \). We say  \( f \) is meromorphic at \( z_0 \) if \( \frac{1}{f} \) is holomorphic near \( z_0 \). 

Another way to say it:

\( f \) is mermorphic at isolated singularity \( z_{0}  \) if \( \exists r>0 \) such that \( \frac{1}{f} \) or \( f \) extends to holomorphic function on \( B_r(z_0) \). 

(this is the old definition we used)
\end{defn}

\begin{thrm}{}{}
If \(f \) is meromorphic near some pole \( z_0 \), then \( \exists r>0  \) such that \( \forall z \in B_{r} (z_0) \) then
\[
    f(z) = \frac{h(z)}{(z - z_0)^n} \ \ n\geq 0
\]
where \( h(z) \) is holomorphic and non-vanishing on \( B_{r} (z_0) \). 
\tcbline
\begin{proof}
By definition \( f \) is holomorphic near \( z_0 \). By the previous theorem we have that \(\frac{1}{f(z)}  = (z - z_0)^n g(z)\) near \( z_0 \). 

\begin{align*}
    f(z) = \frac{\frac{1}{g(z)} }{(z - z_0)^n} = \frac{h(z)}{(z - z_0)^n}.
\end{align*}

\end{proof}

\end{thrm}


 \begin{defn}{Singularity types}{}
 If \( f \) is meromorphic, and we are given some singularity \( z_0 \), if \( n = 0 \) then the singularity is removable, if \( n>0 \) then the singularity is a pole. If \( f \) is not meromorphic then the singularity is essential. 
 \end{defn}
 
\begin{thrm}{Laurent Series}{}
If \( f \) is meromorphic of order \( n \) at \(  z_0 \in \mathbb{C}  \), then \( \exists r>0 \) such that on \( B_{r} (z _0)^{\star }  \coloneqq  B_{r} (z_0) \setminus \left\{ z_0 \right\}   \),
\[
    f(z) = \frac{a_{- n}}{(z - z_0)^n} + \frac{a_{- n + 1}}{(z - z_0)^{n- 1} }+\dots + \frac{a_{- 1}}{(z - z_0)}+ \underbrace{\sum_{n\geq 0}^{\infty}   a_{n} (z - z_0)^n}_{G(z) \text{ holomorphic on } B_{r} (z_0) }
\]

\tcbline

\begin{proof}
\begin{align*}
    f(z) &= \frac{h(z)}{(z - z_0)^{n} } \\
    &= \frac{A_{0}}{(z - z_0)^n} + \frac{A_{1}}{(z - z_0)^{n- 1} }+\dots + \frac{A_{n - 1}}{(z - z_0)}+ \sum_{k\geq 0}^{\infty}   A_{n + k} (z - z_0)^n
\end{align*}
The radius of convergence is the same after the substitution, so we have that \( f \) is holomorphic on the \( B_{r} (z_0) \). 
\end{proof}

\end{thrm}

\begin{defn}{Principal Part and Residue}{}
The principal part of \( f \) at \( z_0 \) is \( a_{- 1}  \). The residue of \( f \) at \( z_0 \) is \( a_ {- 1} = \mathrm{Res}_{z_0}(f)   \). It is important to note that a removable singularity has residue 0. 
\end{defn}

\begin{thrm}{The Residue Formula}{}
Suppose \( f \) holomorphic in a neighborhood of \( \overline{U}  \) except for a finite set of isolated singularities, \( \left\{ z_{i}  \right\} \subset U \). Then 
\[
    \int _{\partial U} f(z) dz = 2\pi i \sum_{j} \mathrm{Res}_{z_{j} } (f). 
\]
\tcbline

\begin{proof}
Apply Cauchy's theorem. 

\begin{align*}
    \int _{\partial U} f(z) dz = \sum_{j} \int _{B_{\epsilon }(z_{j} ) } f(z) dz
\end{align*}
On a neighborhood near the singularities the Laurent series of \(f  \) converges uniformly.

\begin{align*}
    \int _{B_{\epsilon }(z_{j} ) } f(z) dz &= \int _{B_{\epsilon }(z_{j} ) } \sum_{k=- n }^{\infty} a_{k} (z - z_{j} )^k  dz\\
    &=\sum_{k=- n }^{\infty}  \int _{B_{\epsilon }(z_{j} ) } a_{k} (z - z_{j} )^k  dz\\
    &= 2\pi i a_{- 1} = 2\pi i \mathrm{ Res}_{z_{j} }(f) 
\end{align*}

It is important to note that the series uniformly converges, hence why we can exchange the sum and integral. There we have,
\[
    \int _{\partial U} f(z) dz = 2\pi i \sum_{j} \mathrm{Res}_{z_{j} } (f). 
\]
\end{proof}

\end{thrm}

\begin{defn}{Meromorphic Function}{}
\( f \) is meromorphic at a singularity \( z_0 \) if \( \frac{1}{f} \) or \( f \) extends to a holomorphic function on \( B_{r} (z_0) \). Meromorphic functions are preserved by addition, multiplication, and division.  
\end{defn}

\begin{thrm}{}{}
Let \( f \) be holomorphic except at a singularity, and let the limit at that singularity exist. Suppose \( f \) is bounded on \( U  \setminus \left\{ z_0 \right\}  \). Then \( z_{0}  \) is a removable singularity of \( f \).

\tcbline
\begin{proof}
Idea is the if \( f \) extends to  \( z_0 \) as a holomorphic function, then we could have its value by Cauchy's integral formula.

Try extending \( g(z) \) on \( B_{r} (z _0) \) by 
\[
    g(z) = \frac{1}{2\pi i} \int _{\partial B_r(z_0)} \frac{f(w)}{w - z_0} dw.
    \]
We need to show this extension equals the original function and is holomorphic. 

\begin{align*}
    g'(z) &= \frac{d}{dz}\frac{1}{2\pi i} \int _{\partial B_{r} (z_0)}  \frac{f(w)}{(z - z_0)} dw \\
    &= \frac{1}{2\pi i} \int _{\partial B_{r} (z_0)} \frac{f(w)}{(w - z_0)^{2} } dw 
\end{align*}
Since the derivative exists \( g \) is holomorphic. Since the function on the inside is continuous on the boundary as a function of \( z \) we can exchange operators. 
\end{proof}




\end{thrm}


\begin{cor}{}{}
    Let \( f \) be holomorphic in a punctured neighborhood of \( z_0 \). Then \( z_0 \)  is a pole of \( f \) if and only if the limit as the function approaches that point is infinity. 
    \end{cor}

    \begin{thrm}{Casorati - Weierstrass}{}
    Suppose \( f \) is holomorphic on punctured \( B_{r} (z_0) \), and has essential singularity at \( z_0 \). Then \( f(B_{r} (z_0)^ \star ) \) is dense in \( \mathbb{C}  \). 
    \tcbline

    \begin{proof}
    Suppose for the sake of contradiction that \( f(B_{r} (z_0)^\star ) \) is not dense in \C{}. Then \( z_0 \) cannot be an essential singularity. We now prove this:

    Since \( f(B_{r} (z_0)^\star ) \) is not dense, we can find some \( w_0 \in  \mathbb{C} \text{ and } \epsilon >0 \text{ such that } B\epsilon (w_0) \cap f(B_{r} (z_0)^\star ) = \varnothing    \). Observe the following
    \[
        \left\lvert f(z) - w_0 \right\rvert \geq \epsilon \implies \frac{1}{|f(z)- w_0|}\leq \frac{1}{\epsilon } .
    \]
    \end{proof}
    
    \end{thrm}
    
\end{document}